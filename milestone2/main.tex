\documentclass[letterpaper, 10 pt]{article} 

\usepackage{amsmath}
\usepackage{textcomp}
\usepackage{graphicx}
\usepackage[font=footnotesize]{subcaption}
\usepackage[font=footnotesize]{caption}
\usepackage{hyperref}
\usepackage{amssymb}
\usepackage{booktabs}
\usepackage[normalem]{ulem}
\usepackage{verbatim}
\usepackage[export]{adjustbox}
\usepackage{amsmath}
\usepackage{url}
\usepackage{siunitx}
\usepackage[utf8]{inputenc}
\usepackage[TS1,T1]{fontenc}
\usepackage{array, booktabs}
\usepackage{caption}
\usepackage[cal=cm]{mathalfa}

% Labels in IEEE format
\newcommand{\eref}[1]{(\ref{#1})} % Equation
\newcommand{\sref}[1]{Sec.~\ref{#1}} % Section
\newcommand{\figref}[1]{Fig.~\ref{#1}} % Figure
\newcommand{\tref}[1]{Table~\ref{#1}} %Table
\newcommand{\aref}[1]{Algorithm~\ref{#1}} %Algorithm
\renewcommand*\rmdefault{ppl}
\setlength{\textfloatsep}{5pt}

\usepackage{ifthen}
\usepackage[usenames,dvipsnames,table]{xcolor}
\newboolean{include-notes}
\setboolean{include-notes}{true} 
% http://en.wikibooks.org/wiki/LaTeX/Colors
\newcommand{\rhnote}[1]{\ifthenelse{\boolean{include-notes}}%
 {\textcolor{blue}{\textbf{RH: #1}}}{}}

\begin{document}

% paper title
\title{6.857 Final Project: Milestone 2}
\author{Laura Jarin-Lipschitz, Sebastiani Aguirre Navarro and Rachel Holladay}
\maketitle

\section{Algorithms and Evaluation}

(To be written more formally and filled in)

- Our goal is to predict grasp success given a grasp and image
- We will use the dataset provided by big bird\cite{singh2014bigbird} that gives point clouds of objects
- we will use software released by authors of \cite{pas2017grasp} to generate grasps and we will implement the labeling scheme provided in the paper to create the data set
- the data set is quite large so maybe we will only use a portion? (like maybe only a subset of the objects?)

- in the paper they implement four-layer CNN that is the same structure as LeNet~\cite{lecun1998gradient}. "two convolutional/pooling layers followed by one inner product layer with a rectified linear unit at the output and one more inner product layer with a softmax on the output. "
- Input is representation of object surfaces as seen by depth sensor and the grasp candidate. The output is prediction of whether or not the grasp candidate is a grasp. 
- Use stochastic gradient descent with specified learning rate
- They explored the grasping representation, whether the model was trained with CAD models and level of information about object

- however lenet was originally developed for handwritten character recognition, where they define the MNIST data set on that paper. while this problem is still looking for patterns, it is a different application
- therefore in contrast we want to change the CNN structure and learning design choices to see their effect on performance. 
- we will treat what is in the paper as the baseline and compare it against model alterations
- More specifically we will change \rhnote{these edits to the CNN - need to fill some different things we could try}

- in the paper they use several metrics: they compare accuracy as a function of number of training iterations. \rhnote{We will also use this?}
- they also run a series of real robot experiments. as mentioned in milestone 1, we will not be doing this
- \rhnote{We will use what other metrics?}

\section{Risk Management}

(To be more formally written out later)

- We are provided the data set and grasping generation algorithm. We have to implement the labeling algorithm from the paper, which means there is a risk that we make a mistake in our labeling implementation. This would lead to a biased data set, which could break everything. we plan to stress test our labeling procedure and visually inspect (?) the labeling

- we hope to treat the network in the paper as the baseline and, by exploring other architectures, outperform the network. however there is a chance that we cannot. however even if we cant do better we hope to explore the effects of different architectures to better understand. 

- we are going to be testing everything in simulation. This is due to time and resource limitations. However, robotics is inherently about a real robot interacting with the physical world. Therefore, there is a part (from the robotics side, not the ML side) that is incomplete

\section{Division of Labor}

\rhnote{I'm really not sure what to say about this}


{\footnotesize
    \bibliographystyle{ieeetr}
\bibliography{../references}}

\end{document}
