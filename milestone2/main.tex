\documentclass[letterpaper, 10 pt]{article} 

\usepackage{amsmath}
\usepackage{textcomp}
\usepackage{graphicx}
\usepackage[font=footnotesize]{subcaption}
\usepackage[font=footnotesize]{caption}
\usepackage{hyperref}
\usepackage{amssymb}
\usepackage{booktabs}
\usepackage[normalem]{ulem}
\usepackage{verbatim}
\usepackage[export]{adjustbox}
\usepackage{amsmath}
\usepackage{url}
\usepackage{siunitx}
\usepackage[utf8]{inputenc}
\usepackage[TS1,T1]{fontenc}
\usepackage{array, booktabs}
\usepackage{caption}
\usepackage[cal=cm]{mathalfa}

% Labels in IEEE format
\newcommand{\eref}[1]{(\ref{#1})} % Equation
\newcommand{\sref}[1]{Sec.~\ref{#1}} % Section
\newcommand{\figref}[1]{Fig.~\ref{#1}} % Figure
\newcommand{\tref}[1]{Table~\ref{#1}} %Table
\newcommand{\aref}[1]{Algorithm~\ref{#1}} %Algorithm
\renewcommand*\rmdefault{ppl}
\setlength{\textfloatsep}{5pt}

\usepackage{ifthen}
\usepackage[usenames,dvipsnames,table]{xcolor}
\newboolean{include-notes}
\setboolean{include-notes}{true} 
% http://en.wikibooks.org/wiki/LaTeX/Colors
\newcommand{\rhnote}[1]{\ifthenelse{\boolean{include-notes}}%
 {\textcolor{blue}{\textbf{RH: #1}}}{}}

\begin{document}

% paper title
\title{6.857 Final Project: Milestone 2}
\author{Laura Jarin-Lipschitz, Sebastiani Aguirre Navarro and Rachel Holladay}
\maketitle

\section{Algorithms and Evaluation}

Our goal is to use machine learning techniques to predict the probability of successfully grasping an unknown object with a robotic arm. Specifically, we will use a convolutional neural network (CNN) to decide whether or not to attempt a simulated robotic grasp, and then quantify the success of our network's decision in said simulation. The input to our network will be a point cloud generated from a depth sensor and a candidate grasp, and the output will be a probability of grasp success, which we will threshold to decide whether to make a grasp. The training data will consist of point clouds from the dataset Big Bird\cite{singh2014bigbird}, and grasp quality labels created by the open source Grasp Pose Generator(GPG) software\cite{pas2017grasp}. The grasping simulation will be performed in robotics simulator Gazebo [[add citation]] using the model of the Kinova Jaco Robotic arm and gripper [[add github citation]].

In recent work using the Big Bird Data set\cite{pas2017grasp}, a four-layer CNN with architecture near identical to the LeNet\cite{lecun1998gradient} is used. However LeNet was originally developed for handwritten character recognition, and is therefore not likely to be an optimal architecture for this problem. We will explore and compare different CNN architectures to see if we can achieve a higher rate of grasp success as compared to their paper. Given that the dimensions of the images are 1024x1280 in the RGB domain along with 480x640 depth maps, we can test deeper models like Network in Network, Residual Neural Networks or Inception to come up with our own feature extractor or adapt the lower layers of models for object detection that applies similar characteristics as Faster R-CNN, Single Shot Detector or YOLO. The idea is that the lower layers of these object detectors could be adapted to this problem to obtain features related to the object that can be paired with pose features that can be pased to the final classifier layers.
 

\section{Risk Management}

One major risk is that our main goal of improving the architecture of the CNN is not achieved. We hope to treat the network in the paper as the baseline and, by exploring other architectures, outperform the network. however there is a risk that we will not find an architecture with better performance, especially given the difference in our evaluation methods from theirs. However, even if we cannot improve on the percent grasp success in simulation, we hope to explore the effects of different architectures on the types of objects that the network fails to predict correctly. 

Another risk we will is our choice to evaluate our CNN's performance in simulation. We made this choice due to time and resource limitations. However, robotics is inherently about a real robot interacting with the physical world. Therefore, there is a part of the project (from the robotics side, not the ML side) that is incomplete, since simulated grasps are highly dependent on the fidelity of the robot model.

\section{Division of Labor}

\rhnote{I'm really not sure what to say about this}
\rhnote{Since Laura already dug through the code and found the dataset generator, we can leave the generation to her. I can make the input data preprocessing pipeline for the generated dataset and start porting the different architectures into Keras/Tensorflow. What else is there to be done?}


{\footnotesize
    \bibliographystyle{ieeetr}
\bibliography{../references}}

\end{document}
