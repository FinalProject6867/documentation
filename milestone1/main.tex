\documentclass[letterpaper, 10 pt]{article} 

\usepackage{amsmath}
\usepackage{textcomp}
\usepackage{graphicx}
\usepackage[font=footnotesize]{subcaption}
\usepackage[font=footnotesize]{caption}
\usepackage{hyperref}
\usepackage{amssymb}
\usepackage{booktabs}
\usepackage[normalem]{ulem}
\usepackage{verbatim}
\usepackage[export]{adjustbox}
\usepackage{amsmath}
\usepackage{url}
\usepackage{siunitx}
\usepackage[utf8]{inputenc}
\usepackage[TS1,T1]{fontenc}
\usepackage{array, booktabs}
\usepackage{caption}
\usepackage[cal=cm]{mathalfa}

% Labels in IEEE format
\newcommand{\eref}[1]{(\ref{#1})} % Equation
\newcommand{\sref}[1]{Sec.~\ref{#1}} % Section
\newcommand{\figref}[1]{Fig.~\ref{#1}} % Figure
\newcommand{\tref}[1]{Table~\ref{#1}} %Table
\newcommand{\aref}[1]{Algorithm~\ref{#1}} %Algorithm
\renewcommand*\rmdefault{ppl}
\setlength{\textfloatsep}{5pt}

\usepackage{ifthen}
\usepackage[usenames,dvipsnames,table]{xcolor}
\newboolean{include-notes}
\setboolean{include-notes}{true} 
% http://en.wikibooks.org/wiki/LaTeX/Colors
\newcommand{\rhnote}[1]{\ifthenelse{\boolean{include-notes}}%
 {\textcolor{blue}{\textbf{RH: #1}}}{}}

\begin{document}

% paper title
\title{6.857 Final Project: Milestone 1}
\author{Laura Jarin-Lipshitz, Sebastiani Aguirre Navarro and Rachel Holladay}
\maketitle

\rhnote{Below are notes/outline. To be actually written soon}
- grasping is key to robotic manipulation
- our goal is to predict grasp success and quality
- we imagine a scene where there is a robot manipulator and camera that gives view of the object to be grasped
- our goal is to decide how to grasp the object
- so given the view, we want to decide which grasp gives highest probability of success
- generating grasps is a hard robotic problem, so assume we have a black box way to generate grasps
- therefore our input is an image and a grasp and we want to say if success. 

- learning in robotic manipulation has recently gotten really popular. 
- to facilitate large scale learning, people have begin producing datasets
- we have closely investigated two dataset and are (currently) planning to use both: DexNet~\cite{mahler2017dex} and BigBird~\cite{singh2014bigbird}
- DexNet provides ....
- BigBird provides .... Bigbird was used in \cite{pas2017grasp} that develop (this method of creating grasps). 
- DexNet advtanges are comes with grasps and its massive but its synthetic point clouds while big bird is smaller, requires processing but is on real grasps
- Given these dataset hope to apply a CNN architecture, which is the most commonly used method in this field
- we will be evaluating our approach based on our performance on test data and not on a real robotic platform, due to time constrains. 



{\footnotesize
    \bibliographystyle{ieeetr}
\bibliography{../references}}

\end{document}
