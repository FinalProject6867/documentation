\documentclass[letterpaper, 10 pt, conference]{../ieeeconf} 
\IEEEoverridecommandlockouts
\overrideIEEEmargins
\pdfoptionpdfminorversion=4

\usepackage{amsmath}
\usepackage{mathtools}
\usepackage{textcomp}
\usepackage{graphicx}
\usepackage[font=footnotesize]{subcaption}
\usepackage[font=footnotesize]{caption}
\usepackage{hyperref}
\usepackage{amssymb}
\usepackage{booktabs}
\usepackage[normalem]{ulem}
\usepackage{verbatim}
\usepackage[export]{adjustbox}
\usepackage{amsmath}
\usepackage{url}
\usepackage{siunitx}
\usepackage[utf8]{inputenc}
\usepackage[TS1,T1]{fontenc}
\usepackage{array, booktabs}
\usepackage{caption}
\usepackage[cal=cm]{mathalfa}
\usepackage{algorithm}
\usepackage[noend]{algpseudocode}

% Labels in IEEE format
\newcommand{\eref}[1]{(\ref{#1})} % Equation
\newcommand{\sref}[1]{Sec.~\ref{#1}} % Section
\newcommand{\figref}[1]{Fig.~\ref{#1}} % Figure
\newcommand{\tref}[1]{Table~\ref{#1}} % Table
\newcommand{\aref}[1]{Algorithm~\ref{#1}} % Algorithm
\newcommand{\lref}[1]{Line~\ref{#1}} % Line
\renewcommand*\rmdefault{ppl}
\setlength{\textfloatsep}{5pt}

\usepackage{ifthen}
\usepackage[usenames,dvipsnames,table]{xcolor}
\newboolean{include-notes}
\setboolean{include-notes}{true} 
% http://en.wikibooks.org/wiki/LaTeX/Colors
\newcommand{\rhnote}[1]{\ifthenelse{\boolean{include-notes}}%
 {\textcolor{blue}{\textbf{RH: #1}}}{}}
\newcommand{\sanote}[1]{\ifthenelse{\boolean{include-notes}}%
 {\textcolor{green}{\textbf{SAN: #1}}}{}}

\begin{document}

% paper title
\title{6.857 Final Project: Milestone 6}
\author{Sebastiani Aguirre Navarro and Rachel Holladay}
\maketitle

% !TEX root = main.tex

\section{Introduction}
\label{sec:intro}

The ability to grasp objects lies at the heart of robotic manipulation and therefore is fundamental to enabling robots to have complex physical interactions with their environment. 
Grasping a variety of unknown objects is challenging due to sensor and actuator uncertainty and uncertainty with respect to a new object's shape, mass distribution, texture properties, etc. 
Recently, deep neural networks have been used, with significant success, to address these challenges and enable robotic grasping. 

Within the context of this paper, we will make three assumptions with respect to our set-up. 
First, we will be grasping objects from a flat, clutter-free surface, such as an uncrowded table top. 
Second, we assume we have a method of generating \textit{grasp candidates}. 
Last, the robot has either an on-board camera or the environment the robot is operating in has a camera. 
Given an image of the scene captured by the camera, our goal is to evaluate which of these candidate grasps are likely to succeed. 
This creates a binary classification task, where the labels are grasp success and grasp failure. 

During execution, we can imagine that our robot with sample several grasps, execute a grasp that has been predicted to be successful via our classification method.  
Note that since attempting a grasp in a real world setting consumes time and can disrupt the environment, we would prefer to be cautious. 

For our data set we will use the Dexerity Network (DexNet) 2.0 data set, presented in~\cite{mahler2017dex}. 
The data set has 6.7 million grasps definitions, images and analytical grasp metrics, that we further detailed in \sref{sec:data_set}. 
The authors of the data set trained a Grasp Quality Convolutional Neural Network (GQ-CNN), seeen in \figref{fig:dexnet_network}, which achieved 85.7\% accuracy on their binary classification task.

While this is an impressive accuracy rate, we find that it is tied to the fact that nearly 80\% of the data is in the same class. 
When we re-balance the data distribution to have 50\% positive examples and negative examples, their same network achieves only 60\% accuracy. 

\begin{figure}[t!]
    \includegraphics[width=0.99\columnwidth]{figs/dexnet.PNG}
\caption{This is a visualization the GQ-CNN (Grasp Quality Convolutional Neural Network) from \cite{mahler2017dex}. The network takes as input a depth image of the grasp and the distance of gripper to the object and output, after several layers, a prediction of grasp success.} \label{fig:dexnet_network}
\end{figure}

One fundamental assumption when learning is that the training sets and test sets are drawn from the same distribution. 
Maintaining this assumption, we found that this network (and others that we tried), had varying accuracy depending on the distribution. 
In this case, our results affirm the importance of considering the data distribution when evaluating learning algorithms. 

As mentioned, we explored several other architectures, with varying hyper parameters and normalization methods. 
However, we were unable to beat the accuracy rate achieved with the GQ-CNN. 

\textbf{We make the following contributions}:
\begin{enumerate}
    \item We explore the effect of the data set distribution on the accuracy of the model. Specifically, we modify the distribution of the training, validation and testing set and re-evaluate GQ-CNN.
		\item We adapt, train and evaluate three existing networks to our learning problem. 
		\item Using confusion matrices, we analyze the false positive and false negative ratio of our algorithms. From an algorithmic perspective, this helps us better examine the effect of varying data distributions. From a robotic perspective, this informs how efficient the robot might be when utilizing the results of the network. 
\end{enumerate}

We note that our comments with respect to data set distribution can only apply to our specific data set and the models we used. 
To make a more general claim we would need to more exhaustively test across different architectures and different data sets. 
Regardless, we found the effect of the data distribution to be an interesting artifact. 

With respect to the structure of the rest of the paper, we first review related work (\sref{sec:related_work}) and further detail the data set (\sref{sec:data_set}). 
Given our data set, we formally define our problem statement (\sref{sec:problem}) and then explore various data sets using the GQ-CNN (\sref{sec:balance}). 
We explore other architectures (\sref{sec:archs}) and conclude with a brief discussion (\sref{sec:discussion}). 
In particular, we discuss hypotheses on why were we unable to improve our accuracy rate and what makes this problem difficult. 

\begin{comment}

To accomplish the same task, we will be experimenting with new architectures, input formats, other modifications described in~\sref{sec:questions}.
Most of the recent machine learning papers in robotics present a problem, dataset and, usually, an optimized convolutional neural network with some architecture and input format. 
Our goal is to explore the process of finding that CNN and exploring the factors that effect performance. 
While our results will only be verified according to this data set, and therefore cannot be generalized to all CNNs, we hope to gain intution, understanding, and, hopefully, a higher accuracy. 
Having explored various components, we will optimize our final, best architecture. 
\end{comment}

% !TEX root = main.tex

\section{Related Work}
\label{sec:related_work}

\rhnote{pull all papers from bibtex and folder and give summaries}

% !TEX root = main.tex

\section{Data Set}
\label{sec:data_set}

\rhnote{describe dex net data set, very similar to before}

\begin{comment}
We are using the Dex Net 2.0 data set as first presented in~\cite{mahler2017dex}. 
We first briefly summarize their data generation process before describing how we manipulated the data. 

Mahler et al define a generative graphical model defined over the camera pose, object shape and pose, friction coeffient, grasp, depth image and success metric. 
To generate the data set they make i.i.d (independent and identically distributed) samples from their generative graphical model, resulting in 6.7 million data points. 

The data set is defined over 1,500 object meshes that were used in Dex-Net 1.0~\cite{mahler2016dex}, collected from a variety of other data bases and standardized with respect to position.
For each object, they generated 100 parallel jaw grasps via rejection sampling of antipodal pairs and evaluated a grasp metric on each grasp. 
Additionally, each object is paired with a rendered depth image (2.5D point cloud~\footnote{The images are 2D matrics that are referred to as 2.5D in robotics literature because they display depth information.}) from the sampled camera pose. 

The GQ-CNN takes two images as input. The first is the depth image, called the "aligned image", transformed to center and axis align according the grasp point. 
Hence this image captures the scene and grasp in one representation. 
The second image, the "z image" is untransformed and represents the distance from the gripper to the camera.

The data set of 6.7 million data points has 21.1\% positive examples. 
This is unsurprising, since it is much more difficult to find successful grasps, as compared to failed grasps. 

The published Dex-Net 2.0 data set contains both sets of images for each data point in addition to grasp quality metrics and the grasp, represented by a 7-dimensional vector, specifying details of the grasp center, angle, object center and gripper width and several over parameters. 
Our label is given by the robust epsilon quality grasp metric (defined in~\cite{seita2016large}), which is thresholded by the value 0.002 to create binary labels.

From the 6.7 million data points, we create two types of data sets:
\begin{itemize}
    \item \textbf{Unbalanced.} We randomly sample 10,000 data points from our entire set. We expect to sample approximately 20\% positive examples, matching the distribution of the original set.
    \item \textbf{Balanced.} We randomly sample data points until we have 10,000 data points that are 50\% positive examples and 50\% negative examples. 
\end{itemize}
We further discuss the motivation for this distinction in \sref{sec:questions}. 
For all data sets we include all possible features, although some architectures might not leverage all features. 

Since we are sampling our data sets, we will sample multiple copies and average the final results across each version~\footnote{This was not done for this milestone, but will be done in the final report.}.

\end{comment}

% !TEX root = main.tex

\section{Problem Statement and Architectures}
\label{sec:problem_and_archs}

\rhnote{Define input and output. Say using cnn. describe software. Describe one-shot learning. then say we investigated the following changes}
\rhnote{Currently, the input format is a 32x32x1 depth map and a 1x7 pose vector. }

\begin{comment}
\section{Research Questions}
\label{sec:questions}
Below we present several of the research questions we will continue to explore. 


\subsection{Balancing Data Sets}
As mentioned previously, Dex-Net 2.0 contains approximately 20\% positive examples. 
This is not inherently problematic given that the training and testing sets are drawn from the same distribution, with this same ratio.
However, by sampling subsets of our data set, we can achieve any positive-to-negative ratio and thus explore how changing this ratio effects accuracy. 
By doing this, we avoid a model to overfit and artificially think it is doing well by predicting one class on all samples.
(For example, a very trival way to achieve 80\% accuracy would be to predict negative for all examples.)

\subsection{Data Set Size}
The Dex-net data set contains 6.7 million data points. 
For computational reasons, we are sampling a subset of these points. 
However, we can vary the size of this subset to compare the trade-off between the accuracy and the size of the training set. 
Deep learning models tend to require a rather large quantity of data depending the classification task that it is being learned and how complex the data is.

For now, since the input data consists of just small 32x32x1 images and 1x7 vectors, and we learning a binary classification task, we reasoned that we could start by using a small subset of the 6.7 million data points to produce a model of reasonable performance.

\subsection{Architecture Structure}
One of the largest sources of experimentation thus far and continuing forward is our choice of architecture. 
As our starting point, we produced an architecture that follows the concept of residual networks. 
This model consists of instead of learning a direct mapping of the input to the output, to learn ``residue'' over this input. Subsequently, we produced another architecture that follows the Inception model.
Both were described in \sref{sec:results} and have batch normalization after each convolution, to mitigate internal covariances shifts, as well as a dropout rate of 0.7. 
This are initial architectures and not final. 
As we keep training and modifying them accordingly, we will come up with the proper parameters using the validation and testing accuracy as our metric.

\subsection{Normalization}
As seen in our results, we are falling prone to overfitting. 
Therefore, it is critical to explore how we can regularize the system. 

\end{comment}

\section{Results}
\label{sec:results}
Use keras ~\cite{chollet2017keras}
Andreas Network ~\cite{viereck2017learning}

For the following results, we use the balanced dataset with our input as the image for each data point and the 7-dimesional grasp vector. 
During training, 80\% of the dataset was used as train set and the remaining 20\% as test set. 
Below we describe and show the results of two achitectures, which we refer to as the Inception Network~\cite{szegedy2015going} and the ResNet.
As discussed in \sref{sec:questions}, these are the some of the many networks we will be testing.  

The inception network consists of 1 convolution layer in the beginning with 10 filters of size 3x3. 
The output of this layer is passed in parallel to three convolutional layers of sizes 1x1, 3x3, and 5x5, each with 16 filters. 
These outputs are concatenated on the depth dimension and passed through a max pooling layer of 3x3. 
The outputs are flattened with global average pooling and then the pose vector is concatenated before passed to a classifier of one hidden layer of 20 units, as seen in \figref{fig:inception_net}. 

The residual network consists of two convolutional layers, one with 8 filters of size 7x7 and the next with 16 filters of size 3x3.
At this point, the output of this layer branches, such that this same output is passed through two more convolution layers of 32 filters 3x3 and 16 filters 1x1 used as dimension reduction. 
The output of these two layers is added to their input and then passed to another convolution layer of 8 filters of 1x1 for further dimension reduction and then flattened with global average pooling. 
Like for the other network, the pose vector is concatenated to this output before passing it to a classifier with a fully connected layer of 10 hidden units, as shown in \figref{fig:res_net}.
 
In \figref{fig:inception_results}, we can see that for the inception netowrk, the training loss decreases while the validation loss, while oscilating, increases. 
The same trend is shown in \figref{fig:resnet_results} for the residual net. 
This means that, like the results in our previous milestone, the network is overfitting to the data. 
The final training accuracy for inception net and residual net are 70\% and 75\% respectively, while both do 50\% on the validation set. 
One possible modification is to add regularization on the fully connected layers, or to modify the architectures by removing or adding more layers. 
We will continue to explore this as well as experimenting with the representation of the data.

    
\begin{comment}
\begin{figure}[t!]
    \centering
        \includegraphics[width=0.7\columnwidth]{figs/inception_net.png}
    \caption{Network Structure of Inception Net} \label{fig:inception_net}
\end{figure}
\end{comment}

{\footnotesize
    \bibliographystyle{ieeetr}
\bibliography{../references}}

\end{document}
