% !TEX root = main.tex

\section{Introduction}
\label{sec:intro}

\rhnote{tweak introduction from before, expand upon it. add mention of "one shot"}

\begin{comment}
Our goal is to use neural networks to classify whether a particular grasp will succeed on an object.
We utilize the Dexerity Network (DexNet) 2.0 data set~\cite{mahler2017dex}, that has 6.7 million synthetic point clouds with parallel-jaw grasps (a common robot hand type of two parallel fingers) and analytical grasp metrics.
The authors of the data set trained a Grasp Quality Convolutional Neural Network (GQ-CNN), which achieved 85.7\% accuracy on their classification task.
To accomplish the same task, we will be experimenting with new architectures, intput formats, other modifications described in~\sref{sec:questions}.
Most of the recent machine learning papers in robotics present a problem, dataset and, usually, an optimized convolutional neural network with some architecture and input format. 
Our goal is to explore the process of finding that CNN and exploring the factors that effect performance. 
While our results will only be verified according to this data set, and therefore cannot be generalized to all CNNs, we hope to gain intution, understanding, and, hopefully, a higher accuracy. 
Having explored various components, we will optimize our final, best architecture. 

We will first describe the data set generation process and the features provided in the data set~\sref{sec:data_set}. 
Understanding and processing this data set has become a larger element of our project then previously anticipated. 
We next discuss our results thus far~\sref{sec:results}, which are preliminary. 
We will continue to explore these results, as well as our research questions~\sref{sec:questions}.
\end{comment}
