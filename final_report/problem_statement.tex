% !TEX root = main.tex

\section{Problem Statement}
\label{sec:problem}

We now formally define our learning problem. 
We take as input a 32x32x1 depth image and \rhnote{hand vector?}. 

The depth image, called the "aligned image", transformed to center and axis align according the grasp point. 
Hence this image captures the scene and grasp in one representation. 
An example depth image in shown in \figref{fig:depth_image}. \rhnote{add image}. 
We are solving a binary classification problem and hence the output of our network will be 0 or 1 labels. 
A positive label refers to a grasp predicted to be successful and a negative label is a predicted grasp failure. 
Our label in the data set is given by the robust epsilon quality grasp metric (defined in~\cite{seita2016large}), which is thresholded by the value 0.002 to create binary labels.

We split our data into into 80\%, 10\% and 10\% for the training, validation and testing sets respectively.
We measure success by the percentage of correct labels for each set. 

For training we use Keras, an open source neural network library~\cite{chollet2017keras}, that is powered by TensorFlow~\cite{abadi2016tensorflow}. 
